%%%%%%%%%%%%%%%%%
% This is an example CV created using altacv.cls (v1.1.3, 30 April 2017) written by
% LianTze Lim (liantze@gmail.com), based on the 
% Cv created by BusinessInsider at http://www.businessinsider.my/a-sample-resume-for-marissa-mayer-2016-7/?r=US&IR=T
% 
%% It may be distributed and/or modified under the
%% conditions of the LaTeX Project Public License, either version 1.3
%% of this license or (at your option) any later version.
%% The latest version of this license is in
%%    http://www.latex-project.org/lppl.txt
%% and version 1.3 or later is part of all distributions of LaTeX
%% version 2003/12/01 or later.
%%%%%%%%%%%%%%%%

%% If you want to use \orcid or the
%% academicons icons, add "academicons"
%% to the \documentclass options. 
%% Then compile with XeLaTeX or LuaLaTeX.
% \documentclass[10pt,a4paper,academicons]{altacv}

%% Use the "normalphoto" option if you want a normal photo instead of cropped to a circle
% \documentclass[10pt,a4paper,normalphoto]{altacv}

\documentclass[10pt,a4paper]{altacv}

%% AltaCV uses the fontawesome and academicon fonts
%% and packages. 
%% See texdoc.net/pkg/fontawecome and http://texdoc.net/pkg/academicons for full list of symbols.
%% When using the "academicons" option,
%% Compile with LuaLaTeX for best results. If you
%% want to use XeLaTeX, you may need to install
%% Academicons.ttf in your operating system's font %% folder.


% Change the page layout if you need to
\geometry{left=1cm,right=9cm,marginparwidth=6.8cm,marginparsep=1.2cm,top=1cm,bottom=1cm}

% Change the font if you want to.

% If using pdflatex:
\usepackage[utf8]{inputenc}
\usepackage[T1]{fontenc}
\usepackage[default]{lato}
\usepackage[hidelinks]{hyperref}

% If using xelatex or lualatex:
% \setmainfont{Lato}

% Change the colours if you want to
\definecolor{VividPurple}{HTML}{03223C}
\definecolor{SlateGrey}{HTML}{2E2E2E}
\definecolor{LightGrey}{HTML}{666666}
\colorlet{heading}{VividPurple}
\colorlet{accent}{VividPurple}
\colorlet{emphasis}{SlateGrey}
\colorlet{body}{LightGrey}

% Change the bullets for itemize and rating marker
% for \cvskill if you want to
\renewcommand{\itemmarker}{{\small\textbullet}}
\renewcommand{\ratingmarker}{\faCircle}

%% sample.bib contains your publications
\addbibresource{sample.bib}

\begin{document}
\name{Anshu Priya}
\tagline{Indian Institute of Technology, Jodhpur -- 342037}
% Cropped to square from https://en.wikipedia.org/wiki/Marissa_Mayer#/media/File:Marissa_Mayer_May_2014_(cropped).jpg, CC-BY 2.0
%\photo{2.5cm}{mmayer-wikipedia-cc-by-2_0}


\personalinfo{
  
  \email{\href{mailto:priya.1@iitj.ac.in}{priya.1@iitj.ac.in}}
   \phone{(+91) 6350060520}
  \linkedin{\href{https://www.linkedin.com/in/anshu-priya-72b783191/}{Anshupriya}}
   \github{\href{https://github.com/anshu123priya}{Anshupriya}} 
}

%% Make the header extend all the way to the right, if you want.
\begin{fullwidth}
\makecvheader
\end{fullwidth}

%% Provide the file name containing the sidebar contents as an optional parameter to \cvsection.
%% You can always just use \marginpar{...} if you do
%% not need to align the top of the contents to any
%% \cvsection title in the "main" bar.
\cvsection[page1sidebar]{Internship}

\cvevent{Research and Innovation Lab - TCS}{Anatomical landmark detection in 3D CT scan}{May 2020 -- July 2020}{}
\begin{itemize}
%\item Implemented a 3D-Unet network for detection of vertebrae
\item Generated heatmaps for all 26 vertebras from their coordinates and provided isotropic spacing of 2mm to extend its physical dimension
\item Trained on 3D-Unet network to detect the landmark on CT scan and got 9.75mm mean error on testing CT scan
\end{itemize}

\cvevent{MANAS LAB | IIT Mandi | Guide: Dr. Aditya Nigam}{Liver Tumor Segmentation \href{https://github.com/anshu123priya/Liver-Tumor-Segmentation}{\github{}}}{May 2019 -- July 2019}{}
\begin{itemize}
\item Built a segmentation network inspired from UNet having an Hourglass module for its bottleneck
\item Achieved state-of-the-art results (Dice 0.98 for Liver and 0.93 for Liver Tumor) on publicly available datasets of MICCAI 2017 LiTS
\end{itemize}

\cvevent{}{Brain Tumor Segmentation | \href{https://github.com/anshu123priya/Brain-Tumor-Segementation}{\github{}}}{May 2019}{}
\begin{itemize}
\item Build a machine learning algorithm in python that classifies whether a patient
has heart disease or not 
\item Used 14 measuring features like max heart rate, chest pain type, age, sex etc in the publicly available datasets on driven data
\end{itemize}


%%%%%%%%%%%%%%%%%%%%%%% Projects %%%%%%%%%%%%%%%%%

\cvsection[page1sidebar]{Projects}


\cvevent{IIT Jodhpur}{Action Recognition in kids | Guide: Dr. Rajendra Nagar \href{https://github.com/anshu123priya/-Action-recognition-in-kids}{\github{}}}{Feb 2020 -- July 2020}{}
\begin{itemize}
\item Design an algorithm to recognize the various actions of kids like, cruising, crawling, jumping and running
\item Used various deep learning approaches for different input data(skeletal data, dynamic images and RGB videos)
\end{itemize}


\cvevent{}{Sports Equipment Management System | Self Project \href{https://github.com/anshu123priya/Sports-Equipment-Management-System}{\github{}}}{Aug 2019 -- Dec 2019}{}
\begin{itemize}
\item Developed a software system for proper allocation and management of the hostel’s sports equipment
\item Used C++14 for frontend and SQL for backend also used MySql DB on Amazon Web Services
\end{itemize}

\cvevent{}{PathologicAL Myopia Classification | Guide: Dr. Sandeep Kr. \href{https://github.com/anshu123priya/Classification-of-Pathological-Myopia}{\github{}}}{Feb 2019 -- April 2019}{}
\begin{itemize}
\item Classified the fundus photos from PM patients into three labelsPathological Myopia, High Myopia and Normal labels 
\item Achieved 95\% accuaracy by applying data augmentation on publicly available datasets of MICCAI 2019 PALM with base model ResNet
\end{itemize}

\cvevent{}{Bus Management System | Self Project \href{https://github.com/anshu123priya/Sports-Equipment-Management-System}{\github{}}}{Jul 2018 -- Dec 2018}{}
\begin{itemize}
\item Developed a full-featured command line interface application for managing bus ticket booking/canceling  
\item Used C++14 for frontend and file management system for data handling
\end{itemize}


\end{document}
